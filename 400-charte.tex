%http://www.paulla.asso.fr/intranet/relations-avec-partenaires/cda-pau-porte-des-pyrenees/fablab-hackerspace/fonctionnement-pratique-du-fablab/reglement-interieur-work-in-progress
{\small \begin{center}
La version originale est disponible sur le site du M.I.T. : \\ \texttt{http://fab.cba.mit.edu/about/charter/}
\end{center}}
\begin{description}
  \item[Qu'est-ce qu'un FabLab ?]
On désigne sous le nom FabLab un réseau global de laboratoires locaux permettant la création ou fabrication à travers la mise à disposition d'outils de fabrication numériques.

\item[Que trouve-t-on dans un FabLab ?]
Les FabLab proposent un panel d'outils de base permettant de tout fabriquer (ou presque) tout en facilitant le partage des connaissances au sein de ses utilisateurs.

\item[Que propose le réseau des FabLab ?]
Une assistance opérationnelle, éducative, technique, financière et logistique qui va au-delà de ce qu'un seul laboratoire est en mesure de proposer.

\item[Qui peut utiliser les services d'un FabLab ?]
Les FabLab constituent une ressource collective, offrant un accès libre pour les individus ainsi qu'un planning d'activités.

\item[De quoi êtes vous responsable en tant qu'utilisateur ?]
La sécurité : ne rien faire qui puisse blesser un autre utilisateur, vous même ou endommager un outil
Le bon fonctionnement : participer au nettoyage, à la maintenance et tout faire pour améliorer le laboratoire
Le partage des connaissances : contribuer à documenter et enseigner

\item[A qui appartiennent les idées issues du FabLab ?]
Les procédés et créations développés au sein d'un FabLab peuvent être protégés et vendus selon la volonté de leur auteur, mais devraient rester libre d'accès au plus grand nombre afin d'en permettre l'usage et l'étude.

\item[De quelle manière une entreprise peut-elle disposer d'un FabLab ?]
Les activités commerciales peuvent être créées et développées au sein d'un FabLab, mais elle ne doivent pas entrer en conflit avec ses autres activités, elles devraient plutôt se développer à l'extérieur plutôt qu'à l'intérieur du laboratoire et bénéficier au créateurs, laboratoires et réseaux qui ont permis leur succès.
\end{description}
% \vfill
% \begin{center}
%   \includegraphics[width=0.25\textwidth]{fablab-logo.png}
% \end{center}
% \vfill
