Le FabLab de Pau, soutenue par la communauté d'agglomération de Pau\footnote{\texttt{http://www.pau.fr/}} et porté par l'association PauLLA\footnote{\texttt{http://www.paulla.asso.fr/}}, est un espace de fabrication numérique et de prototypage. On y retrouve des machines pilotées par ordinateur ainsi que du matériel permettant la création, l'étude  ou la modification de circuits électroniques. D'autres outils, comme des perceuses ou des machines à coudre peuvent êtres disponibles.

Le FabLab de Pau est également un lieu de rencontre entre créateurs, étudiants, artistes, ingénieurs, particuliers, professionnels, \dots 

Le présent livret a pour objectif de présenter rapidement le FabLab de Pau, ainsi que ses conditions particulières : charte des fablabs, règlement intérieur, horaires d'ouverture ou encore tarifs d'utilisations.

% Présentation en QQOQCCP, simple et complète

\subsection{L'équipe du FabLab}
\subsubsection{Les référents}
% Qui sont les référents, rôles etc
Il existe au sein du FabLab des personnes nommées référents.
Ce sont les référents qui ont en charge l'ouverture et la fermeture du local.
Certains référents sont spécialisés dans l'utilisation et la maintenance d'une ou plusieurs machines.
Les référents sont garants du respect des conditions d'utilisation du FabLab (réglemenent intérieur et charte des fablabs).
Les référents peuvent à leurs tours former d'autres référents.
La liste des référents, leurs aptitudes ainsi que leurs temps de présence au FabLab est disponible sur le site internet du FabLab. %Il faudrait mettre un lien dès que possible : \footnote{\texttt{http://lien}}

\subsubsection{Les partenaires}
% partenaires = personnes morales avec convention paulla ?
Les partenaires sont des structures, par exemple des associations ou des entreprises, ayant signé une convention avec PauLLA pour l'utilisation du FabLab.
Ces partenaires peuvent avoir en leur sein un (ou plusieurs) référent, ce qui leur permet d'utiliser le FabLab en toute autonomie.

\subsection{Horaires d'ouverture}
%Insérer un planning type + ref au site ? adresse web ? ical/vcal ?
Il existe plusieurs types de plages horaires d'ouverture possible au FabLab : les plages reservées pour le projet d'un partenaire, les plages dédiées à un thème, un atelier, ou une formation (en partenariat avec la cyberbase Pau~-~Pyrénées), et les plages d'ouverture sans objectif prédéfini.
Le FabLab est toujours ouvert au public, dans le respect du règlement intérieur, y compris durant les temps "réservés". Néanmoins, l'accès aux machines ne peut être garanti, quelques projets demandant parfois plusieurs heures de temps machine-outil.
Comme spécifié dans le règlement intérieur, le FabLab est susceptible d'être ouvert à toute heure du jour ou de la nuit, n'importe quel jour de l'année.
L'agenda détaillé d'ouverture est disponible sur le site internet du FabLab. %Il faudrait mettre un lien dès que possible : \footnote{\texttt{http://lien}}

\subsection{Lieu}
Le FabLab de Pau est situé 4 rue Despourrins, à Pau, au dessus de la cyberbase Pau~-~Pyrénées
Un plan d'accès est visible à la fin de ce livret.
L'accès au FabLab :
\begin{itemize}
  \item en vélo : des supports d'attache pour les vélos sont disposés devant l'entrée de la cyberbase, à quelques mètres du FabLab. Une station "IDECycle" est située rue Jean Monnet, face au centre commercial, à environ 200 m du FabLab.
  \item en autobus : le "Pôle Bosquet" est une gare routière. Elle est desservie par de nombreuses lignes, y compris la navette gratuite "Coxitis". Un plan détaillée des lignes est disponibles sur le site du réseau idelis\footnote{\texttt{http://reseau-idelis.com/upload/plans/plans\_2013/pole\_Bosquet.pdf}}. Depuis la gare SNCF de Pau, la ligne T2 (direction "Centre Hospitalier"), est en 7 minutes au quai B du Pôle Bosquet, situé face au FabLab.
  \item en voiture : il n'y a pas de parking au FabLab. Il est possible de se garer dans les rue adjacentes ou au parking souterrain "Centre Bosquet", à environ 300 m du FabLab.
\end{itemize}
%Une photo ?
Pour entrer au FabLab, il faut sonner (porte B). Le FabLab se situe au premier étage. %FIXME : porte B ?
Le FabLab est accessible aux personnes à mobilité réduite.
Conformément à la réglementation concernant les établissements recevant du public, le nombre maximal de personnes présentes dans le local du FabLab est de 19. Les référents se gardent le droit de refuser des entrées si ce nombre est atteint.
La propreté et le rangement du FabLab incombe à chaque utilisateur, en accord avec la charte des fablabs.

\subsection{Que fait-on au FabLab ?}
%Oui, oui, on bricole...
Le FabLab de Pau permet le prototypage d'objets, des créations originales ou des améliorations des objets du quotidien.
Bien qu'il n'est pas souhaitable que le FabLab soit un "service après vente", la réparation est possible, par exemple si cette réparation n'est pas proposée au public ou si la pièce de rechange n'existe plus.
Afin de mener à bien les différents projets, de nombreuses ressources sont disponibles : fers à souder, composants électroniques, \textit{breadboards}, cartes Arduino, mais aussi une imprimante 3D et une découpeuse laser.

Le FabLab est également un lieu de rencontre : on peut y venir simplement pour partager des idées ou apprendre, sans projets précis prédéfinis. La mixité culturel présente au sein du FabLab permet ainsi la rencontre de personnes provenant d'horizons très différents, comme celle d'un électronicien et d'un artiste peintre.

En partenariat avec la cyberbase Pau~-~Pyrénées, des ateliers de formation sont prévus : modélisation 3D, électronique, programmation python, \dots
Pour plus d'information sur le programme du FabLab, se référer à l'agenda.

\subsection{Comment fonctionne le FabLab ?}

\subsection{Les tarifs du FabLab}
% Trésorier, je compte sur toi pour cette section

\subsubsection{Utilisation des machines}

\subsection{Pourquoi venir au FabLab ?}
%Motivons les à venir !
