Le FabLab de Pau, soutenue par la communauté d'agglomération de Pau\footnote{\texttt{http://www.pau.fr/}} et porté par l'association PauLLA\footnote{\texttt{http://www.paulla.asso.fr/}}, est un espace de fabrication numérique et de prototypage. On y retrouve des machines pilotées par ordinateur ainsi que du matériel permettant la création, l'étude  ou la modification de circuits électroniques. D'autres outils, comme des perceuses ou des machines à coudre peuvent êtres disponibles.

Le FabLab de Pau est également un lieu de rencontre entre créateurs, étudiants, artistes, ingénieurs, particuliers, professionnels, \dots 



Le présent livret a pour objectif de présenter rapidement le FabLab de Pau, ainsi que ses conditions particulières : charte des fablabs, règlement intérieur, horaires d'ouverture ou encore tarifs d'utilisations.

% Présentation en QQOQCCP, simple et complète

\subsection{L'équipe du FabLab}
\subsubsection{Les référents}
% Qui sont les référents, rôles etc
\subsubsection{Les partenaires}
% partenaires = personnes morales avec convention paulla ?
\subsection{Horaires d'ouverture}
%Insérer un planning type + ref au site ? adresse simple ? ical/vcal ?
\subsection{Lieu}
Le FabLab de Pau est situé 4 rue Despourrins, à Pau, au dessus de la cyberbase. Non loin du centre ville
%Ajouter les accès bus, parking bosquet, attache vélo près de la porte de la cyberbase 
%Une photo ?
\subsection{Que fait-on au FabLab ?}
%Oui, oui, on bricole...
\subsection{Comment fonctionne le FabLab ?}

\subsection{Les tarifs du FabLab}
% Trésorier, je compte sur toi pour cette section

\subsubsection{Utilisation des machines}

\subsection{Pourquoi venir au FabLab ?}
%Motivons les à venir !
