%http://www.paulla.asso.fr/intranet/relations-avec-partenaires/cda-pau-porte-des-pyrenees/fablab-hackerspace/fonctionnement-pratique-du-fablab/charte-du-fablab

L´idée des fab labs provient du M.I.T., qui a créé une charte d´utilisation utilisé dans le monde entier. En voici la version française :\footnote{Disponible sur \texttt{http://fablab.fr/projects/project/charte-des-fab-labs/}}

\paragraph{Mission} : les fab labs sont un réseau mondial de laboratoires locaux, qui rendent possible l'invention en ouvrant aux individus l'accès à des outils de fabrication numérique.
 
\paragraph{Accès} : vous pouvez utiliser le fab lab pour fabriquer à peu près n'importe quoi (dès lors que cela ne nuit à personne) ; vous devez apprendre à le fabriquer vous-même, et vous devez partager l'usage du lab avec d'autres usages et utilisateurs.
 
\paragraph{Éducation} : la formation dans le fab lab s'appuie sur des projets et l'apprentissage par les pairs ; vous devez prendre part à la capitalisation des connaissances à et à l'instruction des autres utilisateurs.
 
\paragraph{Responsabilité} : vous êtes responsable de :
\begin{itemize}
  \item La sécurité : Savoir travailler sans abimer les machines et sans mettre en danger les autres utilisateurs ;
  \item La propreté : Laisser le lab plus propre que vous ne l'avez trouvé ;
  \item La continuité : Assurer la maintenance, les réparations, la quantité de stock des matériaux, et reporter les incidents ;
\end{itemize}

\paragraph{Secret} : les concepts et les processus développés dans les fab labs doivent demeurer utilisables à titre individuel. En revanche, vous pouvez les protéger de la manière qui vous choisirez.
 
\paragraph{Business} : des activités commerciales peuvent être incubées dans les fab labs, mais elles ne doivent pas faire obstacle à l'accès ouvert. Elles doivent se développer au-delà du lab plutôt qu'en son sein et de bénéficier à leur tour aux inventeurs, aux labs et aux réseaux qui ont contribué à leur succès.
\vfill
\begin{center}
  \includegraphics[width=0.25\textwidth]{fablab-logo.png}
\end{center}
\vfill
