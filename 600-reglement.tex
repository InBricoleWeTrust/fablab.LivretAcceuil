%http://www.paulla.asso.fr/intranet/relations-avec-partenaires/cda-pau-porte-des-pyrenees/fablab-hackerspace/fonctionnement-pratique-du-fablab/reglement-interieur-work-in-progress

\subsection*{Article 1. Préambule}
Le FabLab est un lieu ouvert situé à Pau. Il se trouve 4 rue Despourrins, au premier étage, et est géré par l'association PauLLA. Son objectif est d'être un "Laboratoire de Fabrication" qui s'inscrit dans le cadre défini par le M.I.T (\textit{Massachussetts Institute of Technology}) dans la charte du même nom.
Le présent règlement, rédigé et validé par l'association PauLLA, fixe les droits et obligations des usagers du FabLab. Les représentants de l'association sont chargés de le faire appliquer.

\subsection*{Article 2. Champ d'application}
Quiconque souhaite fréquenter le FabLab s'engage à accepter les articles du présent règlement qui s'applique à toute personne présente dans son enceinte et/ou dans les activités qui pourraient être organisées à l'extérieur.

\subsection*{Article 3. Conditions d'accès}
\subsubsection*{Section 3.1 Horaires}
Le FabLab est susceptible d'être ouvert à n'importe quelle heure de la journée entre 0h et 23h59 du 1er janvier au 31 décembre. Les créneaux d'ouverture sont annoncés sur son site internet, via son compte Twitter, sur son canal IRC et toute autre solution technique que l'association PauLLA jugera utile de mettre en place.
\subsubsection*{Section 3.2 Tarifs}
L'accès au FabLab est gratuit. Les tarifs liés à l'utilisation des machines, aux consommables, à la participation à certaines animations spécifiques sont affichés dans le local et sur le site internet. Ils sont fixés par l'association PauLLA et peuvent être révisés sans preavis.
\subsubsection*{Section 3.3 Accès des mineurs}
Les mineurs souhaitant fréquenter le FabLab doivent obtenir l'autorisation d'un représentant légal. Les enfants de moins de 12 ans devront systématiquement être accompagnés d'un adulte. Pour la participation aux activités, ils devront être amenés et récupérés par la même personne.

\subsection*{Article 4. Fonctionnement}
\subsubsection*{Section 4.1 Les référents}
Lorsque le FabLab est ouvert, les référents de l'association PauLLA sont à la disposition des usagers pour renseigner à propos du fonctionnement du lieu, initier à l'utilisation des machines, échanger des connaissances... etc.
\subsubsection*{Section 4.2 Accès aux machines}
Certaines machines mises à la disposition des utilisateurs peuvent présenter un danger si elles sont mal utilisées. Les personnes désirant les utiliser sont invitées à respecter les consignes affichées à proximité. En cas de doute, ils sont tenus de s'adresser à un référent.

\subsection*{Article 5. Règles générale d'utilisation}
\subsubsection*{Section 5.1 Interdictions}
Il est interdit aux usagers :
\begin{itemize}
  \item De fumer et de vapoter dans le local.
  \item De manger et de boire en dehors de la zone prévue à cet effet
  \item De se mettre en danger ou de mettre en danger les autres utilisateurs par une utilisation inappropriée des outils mis à leur disposition
\end{itemize}
\subsubsection*{Section 5.2 Comportement des usagers}
Les usagers sont tenus de respecter le calme à l'intérieur des locaux et d'y avoir une tenue correcte. Ils ne devront en aucun cas être cause de nuisances pour les autres usagers et les référents.
Les effets personnels des usagers sont placés sous leur propre responsabilité.
Toute dégradation des locaux ou du matériel, tout vol, pourra entraîner des sanctions internes et/ou une poursuite judiciaire impliquant notamment la réparation du dommage.
Toute agression physique ou verbale à l'encontre des référents pourra faire l'objet d'une plainte.
\subsubsection*{Section 5.3 Hygiène et sécurité}
Il est interdit de pénétrer ou de demeurer dans le FabLab en état d'ébriété ou sous l'emprise de drogues.
L'accès des animaux est interdit dans le Fablab à l'exception des animaux guides.

\subsection*{Article 6. Limitation du droit d'usage}
Les référents du FabLab sont chargés de mettre en œuvre les moyens techniques et humains visant à faire appliquer le présent règlement sous l'autorité du Président de l'association PauLLA.
Le non-respect d'une ou plusieurs des consignes énoncées ci-dessus entraînera les sanctions suivantes :
\begin{itemize}
  \item éviction des lieux sur le champ
  \item interdiction temporaire ou définitive d'accès au FabLab
\end{itemize}

\subsection*{Article 7. Modification du présent règlement}
Le présent règlement pourra être modifié à tout moment, les modifications s'appliqueront de plein droit à tout usager.
