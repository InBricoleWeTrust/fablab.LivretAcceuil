\paragraph{Au départ}
\begin{itemize}
  \item Nom du projet : 
  \item Auteur(s) : 
  \item Description succincte : [quelques lignes descriptives du projet tel que son auteur l'imagine]
  \item Date de début :
  \item Date de fin estimée : 
\end{itemize}
\paragraph{Projet en cours}
~\\
Statut du projet : Démarrage / En cours de réalisation / Terminé / Abandonné
\subparagraph{Réalisation : }
\textit{Veuillez détailler ici les grandes étapes de la réalisation, en tous cas ce qui est fait au FabLab. Ne pas lésiner sur la quantité de texte et de photos.
Merci de mettre en ligne les fichiers permettant à d'autres utilisateurs de reprendre le projet en cours ou de le réaliser à l'identique en précisant les logiciels utilisés.
La liste des outils et machines ayant permis au projet de se faire est également la bienvenue.
Si votre projet est basé sur une réalisation existante, merci d'en indiquer les références (lien web, article de presse\dots).}

\paragraph{Projet terminé}
~\\
Date de fin réelle : \\
\textit{Le projet que vous avez mené :
\begin{itemize}
  \item est-il arrivé à son terme ?
  \item le résultat final correspond-il à ce qui a été envisagé au départ ?
  \item veuillez indiquer une estimation du temps passé
  \item avez-vous trouvé au FabLab tous les outils permettant de mener à bien votre projet ? Si la réponse est non, qu'est-ce qui vous a manqué ?
  \item avez-vous eu l'occasion d'échanger des connaissances / des bonnes pratiques / des idées nouvelles avec les autres utilisateurs du FabLab ?
\end{itemize}}
